\documentclass[]{report}   % list options between brackets
\usepackage{}              % list packages between braces

% type user-defined commands here

\begin{document}

%%%%%%%%%%% Title page %%%%%%%%%%%

\title{Solid Object Visualisation Case Study}
%\subtitle{CS32310 Assignment Report}

\author{Hristoz Stefanov Stefanov}
\date{\today}

\maketitle





%%%%%%%%%%%% Abstract %%%%%%%%%%%%

\begin{abstract}

%> SEG instruction to authors <%
\begin{quotation}
	``Please pay particular attention to the preparation of your abstract; use the material in this reference as a
	guide. Every manuscript other than a discussion must be accompanied by an informative abstract of no more than
	one paragraph (200 to 300 words). The abstract should be self-contained. No references, figures, tables, or
	equations are allowed in an abstract. Do not use new terminology in an abstract unless it is defined or is well
	known from prior publications. SEG discourages the use of commercial names or parenthetical statements. 
	The abstract must not simply list the topics covered in the paper but should (1) state the scope and principal
	objectives of the research, (2) describe the methods used, (3) summarize the results, and (4) state the principal
	conclusions. Do not refer to the paper itself in the abstract. For example, do not say, ``In this paper, we will
	discuss�''

	The abstract must stand alone as a very short version of the paper rather than as a description of the contents.
	Remember that the abstract will be the most widely read portion of the paper. Various groups throughout the world
	publish abstracts of Geophysics papers. Readers and occasionally even reviewers may be influenced by the abstract
	to the point of final judgment before the body of the paper is read.''
\end{quotation}
%> SEG instruction to authors <%


Blah blah blah

\end{abstract}





%%%%%%%%% 1. Introduction %%%%%%%%

\chapter{Introduction}		% chapter 1


%> SEG instruction to authors <%
\begin{quotation}
	``The purpose of the introduction is to tell readers why they should want to read what follows the introduction.
	This section should provide sufficient background information to allow readers to understand the context and
	significance of the problem. This does not mean, however, that authors should use the introduction to rederive
	established results or to indulge in other needless repetition. The introduction should (1) present the nature
	and scope of the problem; (2) review the pertinent literature, within reason; (3) state the objectives; (4)
	describe the method of investigation; and (5) describe the principal results of the investigation.''
\end{quotation}
%> SEG instruction to authors <%


Blah blah blah





%%%%%%%%%%% 2. Methods %%%%%%%%%%%
\chapter{Methods}           % chapter 2

%> SEG instruction to authors <%
\begin{quotation}
	``The methodology employed in the work should be described in sufficient detail so that a competent geophysicist
	could duplicate the results. More detailed items (e.g., heavy mathematics) often are best placed in appendices.
	For complex mathematical articles, authors are strongly encouraged to include a table of symbols.''
\end{quotation}
%> SEG instruction to authors <%

Blah blah blah


%----------- 2.1. Shift ----------
\section{Shift}

Blah blah blah


%----------- 2.2. Scale ----------
\section{Scale}

Blah blah blah


%---------- 2.3. Rotate ----------
\section{Rotate}

Blah blah blah


%---------- 2.4. Reflect ---------
\section{Reflect}

Blah blah blah


%------ 2.5. Parallel Proj. ------
\section{Parallel Projection}

Blah blah blah


%----- 2.6. Perspective Proj. ----
\section{Perspective Projection}

Blah blah blah





%%%%%%%%%%% 3. Results %%%%%%%%%%%
\chapter{Results}           % chapter 3

%> SEG instruction to authors <%
\begin{quotation}
	``The results section contains applications of the methodology described above. The results of experiments
	(either physical or computational) are data and can be presented as tables or figures and analyses. Whenever
	possible, include at least one example of recorded data to illustrate the technology or concept being proposed.
	Case-history results are usually geologic interpretations.

	Selective presentation of results is important. Redundancy should be avoided, and results of minor variations on
	the principal experiment should be summarized rather than included. Details appearing in figure captions and
	table heads should not be restated in the text. In a well-written paper, the results section is often the
	shortest.''
\end{quotation}
%> SEG instruction to authors <%


Blah blah blah





%%%%%%%%%% 4. Conclusion %%%%%%%%%
\chapter{Conclusion}		% chapter 4

%> SEG instruction to authors <%
\begin{quotation}
	``The conclusion section should include (1) principles, relationships, and generalizations inferred from the
	results (but not a repetition of the results); (2) any exceptions to or problems with those principles,
	relationships, and generalizations, as indicated by the results; (3) agreements or disagreements with previously
	published work; (4) theoretical implications and possible practical applications of the work; and (5) conclusions
	drawn (especially regarding significance). In particular, with reference to item (1) above, a conclusion that
	only summarizes the results is not acceptable.''
\end{quotation}
%> SEG instruction to authors <%


Blah blah blah






%%%%%%%%%% Bibliography %%%%%%%%%%
\begin{thebibliography}{9}
  % type bibliography here
\end{thebibliography}





\end{document}